\documentclass{article}

\usepackage{a4}
\usepackage[utf8]{inputenc}
\usepackage{listings}
\usepackage{graphicx}
\usepackage{url}

\title{A short guide to {\bf user preferences}\\ for the MISC dependency solvers}
\author{Roberto Di Cosmo\\Univ Paris Diderot, Sorbonne Paris Cit\'e,\\ PPS, UMR 7126, CNRS,
    F-75205 Paris, France\\\url{roberto@dicosmo.org}}
\date{}

\begin{document}
\maketitle

\section{The basics}

When performing an installation or an upgrade, the user of a package manager 
explicitly formulates a \emph{user request} mentioning the packages that need
to be installed or upgraded, possibly with same constraints (e.g. to
install package \texttt{foo} in some version greater then \texttt{3.2-5}).

The user may also be concerned about the packages that are changed, the packages
that are not up to date, the packages that get removed, or even more exotic
features like “the overall installed size” of the solution proposed by the
package manager. This is possible when using package managers designed
according to the principles stated in~\cite{DepSolversNP2012}, and
using the dependency solvers that participated in the latest MISC competition
\footnote{http://www.mancoosi.org/misc/}.

These solvers are all able to read the CUDF format~\cite{cudf2008} and provide
a specialised \emph{preference language} that allows to express these high-level
concerns.  A \emph{preference expression} is built from three basic ingredients:
\begin{description}
  \item[package selectors] a predefined set of keywords allow to identify in a proposed solution
                           certain classes of packages (the ones that changed, the ones that got removed, etc.)
  \item[measurements] a predefined set of measurement functions can be appliead to a package selector
                      to obtain an integer value (the number of package selected, the number of packages selected that are up-to-date, etc.)
  \item[maximisation/minimisation] one can ask the solver to find a solution that maximises or minimises the value of a measurement 
  \item[aggregation] one can ask the solver to aggregate several maximised or minimised measurements in lexicographical order
\end{description}

We summarise in what follows the package selectors and the measurements available in the latest preference language.

\section{Package selectors}

In the following we describe the package available selectors. Notice that in what follows
$S$ stands for the solution proposed by the solver, $I$ stands for the packages installed
before calling the solver, $IntReq$ stands for the part of the user request specifying
the packages to install, and $UpgReq$ stands for the part of the user request specifying
the packages to upgrade.

\begin{description}
  \item[solution] denotes the set $S$ of packages that are in the solution found by the solver
  \item[changed] denotes $\{p \in S ~\mid~  p \not\in I\} \cup \{p \in I ~\mid~  p
    \not\in S\}$, in other words the symmetric set difference between S and
    I. That is, changed denotes the set of packages that have been newly
    installed or removed. Note that when a package with name foo gets upgraded
    from version 1 to version 2 this means that (foo,1) gets removed and
    (foo,2) gets installed, and as a consequence the set changed contains both
    packages (foo,1) and (foo,2).
  \item[new] denotes $\{ p \in S ~\mid~  p.name \not\in I.name \}$ is the set of
    packages that are finally installed, and for which no package of the same
    name was initially installed.
  \item[removed] denotes $\{ p \in I ~\mid~  p.name \not\in S.name \}$ is the set of
    packages that were initially installed, and for which no package of the
    same name is finally installed.
  \item[up] denotes $\{ p \in S ~\mid~  \exists q \in I \mbox{ with } q.name = p.name, \mbox{ and } \forall q \in I
    \mbox{ with } q.name = p.name : q.version < p.version\}$. 
    In other words, up is the set of upgraded packages. In case multiple
    versions of packages with the same name may be installed, the upgrade
    consists of the packages with a version higher than any previously
    installed version.
  \item[down] denotes $\{ p \in S ~\mid~  \exists q \in I \mbox{ with } q.name = p.name, \mbox{ and } \forall q \in I
    \mbox{ with } q.name = p.name : q.version > p.version\}$. 
    In other words, up is the set of downgraded packages. In case multiple
    versions of packages with the same name may be installed, the downgrade
    consists of the packages with a version higher than any previously
    installed version.
\end{description}

Starting from version \texttt{1.9.1}, the \texttt{aspcud} solver also supports the following additional selectors

\begin{description}
  \item[installrequest] denotes $\{ p\in S ~\mid~  p \mbox{ mentioned in } InstReq\}$, i.e. the packages in the solution
    that the user explicitly requested to install
  \item[upgraderequest] denotes $\{ p\in S ~\mid~  p \mbox{ mentioned in } UpgReq\}$, i.e. the packages in the solution
    that the user explicitly requested to upgrade
  \item[request] is the union of the above two
\end{description}

\section{Measurements}

On each of package selector $X$, we can ask the solver to compute one of the following measures:

\begin{description}
\item[count($X$)] is the number of packages in $X$
\item[sum($X$,f)] is the sum over all packages in the set $X$ of the value of their \emph{integer} field $f$.
                A typical example of this would be \texttt{sum(solution,size)}, with \texttt{size} an 
                integer-valued property of packages indicating their size.
\item[notuptodate($X$)] is the number of packages in $X$ that are not in the latest known version.
\item[unsat\_recommends($X$)] counts the number of disjunctions in Recommends-fields of packages in $X$ that are not satisfied by $S$.
\item[aligned($X$,g1,g2)] is formally $card(\{ (x.g1,x.g2) \mid x \in X \}) - card(\{ x.g1 \mid x \in X \})$.
        In other words, we first cluster the packages in X according to their values at the properties g1 and g2 and count the number of clusters, yielding a value $v1$. Then we do the same when clustering only by the property g1, yielding a value $v2$. The value returned is then $v1-v2$. 
This measure can be used to count \emph{version changes} as defined in~\cite{AlignedUpgrades2011}.
\end{description}
      
\section{Optimising and combining measurements}

The measures described above are used as functions that the solver is the asked to maximise or minimise.
In the preference language, maximisation is specified by prepending the plus sign $+$ to a measure, and
minimisation is specified by prepending a minus sign $-$ to a measure. For example, \texttt{-count(changed)}
requires a solution that minimises changes with respect to the initial configuration, while \texttt{+count(up)}
requests a solution maximising the upgrades.

Several criteria can be aggregated together in lexicographical order, so for
example \texttt{-count(removed),-count(changed)} is a preference expression
that specifies a solution which minimises the number of removed packages,
and then minimises the number of changed ones.

\section{Older preference language}

Solvers participating in earlier versions of the MISC competition supported a simpler preference language,
that can be directly translated into the more sophisticated one described above.\\

The correspondence is given below.\\

\begin{center}
\begin{tabular}{|c|c|}
\hline
Old language & New language \\\hline
removed     & count(removed) \\
new         & count(new) \\
changed     & count(changed) \\
notuptodate & notuptodate(solution)\\
unsat\_recommends & unsat\_recommends(solution)\\\hline
\end{tabular}
\end{center}


\section{Package managers and solvers supporting user preferences}

At the time of this writing, here is the list of available dependency solvers supporting
the different preference languages:

\begin{description}
 \item[full selectors and measures] \mbox{}\\\texttt{aspcud}\footnote{\url{http://sourceforge.net/projects/potassco/files/aspcud/}} versions \texttt{1.9.1} and later
 \item[MISC 2012 selectors and measures]\mbox{}\\
      \texttt{aspcud} in versions \texttt{1.8} and later;\\
      \texttt{p2cudf} \footnote{\url{http://wiki.eclipse.org/Equinox/p2/CUDFResolver}}
 \item[older preference language]\mbox{}\\
      \texttt{aspcud} versions \texttt{1.9.1} and later,\\
      \texttt{aspcud} versions before \texttt{1.8} (e.g. version \texttt{2011.03.17} in Debian and Ububntu);\\
      \texttt{mccs}\footnote{\url{http://www.i3s.unice.fr/~cpjm/misc/mccs.html}},\\
      and \texttt{packup}\footnote{\url{http://sat.inesc-id.pt/~mikolas/sw/packup/}}.
\end{description}

These solvers can be used by recent versions of the \texttt{apt-get} and \texttt{opam} package managers.

\bibliography{dicosmo,biblio}
\bibliographystyle{abbrv}

\end{document}
