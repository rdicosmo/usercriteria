\documentclass{article}

\usepackage{a4wide}
\usepackage[utf8]{inputenc}
\usepackage{listings}
\usepackage{graphicx}
\usepackage{url}
\usepackage{color}
\usepackage{alltt}
\usepackage{gitinfo}
\usepackage{backnaur}

% versioning
\newcommand{\VERSION}{Revision \gitVtags: \gitAbbrevHash{}}
\newcommand{\DATE}{\gitAuthorDate}

\newcommand{\solverpref}[1]{{\texttt{#1}}}
\newcommand{\proposal}[1]{{\textcolor{blue}{#1}}}
\newcommand{\aspcudext}[1]{{\textcolor{green}{#1}}}
\newcommand{\bslash}{\tt\char`\\}

\title{A short guide to {\bf user preferences}\\ for the MISC dependency solvers\\ and a proposal for its extension}
\author{Roberto Di Cosmo\\Univ Paris Diderot, Sorbonne Paris Cit\'e,\\ PPS, UMR 7126, CNRS,
    F-75205 Paris, France\\\url{roberto@dicosmo.org}}
\date{\DATE{} - \VERSION{}}

\begin{document}
\maketitle

\section{The basics}

When performing an installation or an upgrade, the user of a package manager 
explicitly formulates a \emph{user request} mentioning the packages that need
to be installed or upgraded, possibly with same constraints (e.g. to
install package \texttt{foo} in some version greater then \texttt{3.2-5}).

The user may also be concerned about the packages that are changed, the packages
that are not up to date, the packages that get removed, or even more exotic
features like “the overall installed size” of the solution proposed by the
package manager. This is possible when using package managers designed
according to the principles stated in~\cite{DepSolversNP2012}, and
using the dependency solvers that participated in the latest MISC competition
\footnote{http://www.mancoosi.org/misc/}.

These solvers are all able to read the CUDF format~\cite{mancoosi-tr3} and provide
a specialised \emph{preference language} that allows to express these high-level
concerns.  A \emph{preference expression} is built from four basic ingredients:
\begin{description}
  \item[package selectors] a predefined set of keywords allow to identify in a proposed solution
                           certain classes of packages (the ones that changed, the ones that got removed, etc.)
  \item[measurements] a predefined set of measurement functions can be appliead to a package selector
                      to obtain an integer value (the number of package selected, the number of packages selected that are up-to-date, etc.)
  \item[maximisation/minimisation] one can ask the solver to find a solution that maximises or minimises the value of a measurement 
  \item[aggregation] one can ask the solver to aggregate several maximised or minimised measurements in lexicographical order
\end{description}

We summarise in what follows the package selectors and the measurements available in the latest preference language.

\section{Package selectors}

In the following we describe the available \emph{package selectors}. Notice that in what follows
$S$ stands for the solution proposed by the solver, $I$ stands for the packages installed
before calling the solver, $IntReq$ stands for the part of the user request specifying
the packages to install, and $UpgReq$ stands for the part of the user request specifying
the packages to upgrade.

\begin{description}
  \item[solution] denotes the set $S$ of packages that are in the solution found by the solver
  \item[changed] denotes $\{p \in S ~\mid~  p \not\in I\} \cup \{p \in I ~\mid~  p
    \not\in S\}$, in other words the symmetric set difference between S and
    I. That is, changed denotes the set of packages that have been newly
    installed or removed. Note that when a package with name foo gets upgraded
    from version 1 to version 2 this means that (foo,1) gets removed and
    (foo,2) gets installed, and as a consequence the set changed contains both
    packages (foo,1) and (foo,2).
  \item[new] denotes $\{ p \in S ~\mid~  p.name \not\in I.name \}$, the set of
    packages that are finally installed, and for which no package of the same
    name was initially installed.
  \item[removed] denotes $\{ p \in I ~\mid~  p.name \not\in S.name \}$, the set of
    packages that were initially installed, and for which no package of the
    same name is finally installed.
  \item[up] denotes $\{ p \in S ~\mid~  \exists q \in I \mbox{ with } q.name = p.name, \mbox{ and } \forall q \in I
    \mbox{ with } q.name = p.name : q.version < p.version\}$. 
    In other words, up is the set of upgraded packages. In case multiple
    versions of packages with the same name may be installed, the upgrade
    consists of the packages with a version higher than any previously
    installed version.
  \item[down] denotes $\{ p \in S ~\mid~  \exists q \in I \mbox{ with } q.name = p.name, \mbox{ and } \forall q \in I
    \mbox{ with } q.name = p.name : q.version > p.version\}$. 
    In other words, up is the set of downgraded packages. In case multiple
    versions of packages with the same name may be installed, the downgrade
    consists of the packages with a version higher than any previously
    installed version.
\end{description}

Starting from version \texttt{1.9.1}, the \texttt{aspcud} solver~\cite{aspcud-2011} also supports the following additional selectors

\begin{description}
  \item[installrequest] denotes $\{ p\in S ~\mid~  p \mbox{ mentioned in } InstReq\}$, i.e. the packages in the solution
    that the user explicitly requested to install
  \item[upgraderequest] denotes $\{ p\in S ~\mid~  p \mbox{ mentioned in } UpgReq\}$, i.e. the packages in the solution
    that the user explicitly requested to upgrade
  \item[request] is the union of the above two
\end{description}

\section{Measurements}

On each package selector $X$, we can ask the solver to compute one of the following measures:

\begin{description}
\item[count($X$)] is the number of packages in $X$
\item[sum($X$,f)] is the sum over all packages in the set $X$ of the value of their \emph{integer} field $f$.
                A typical example of this would be \texttt{sum(solution,size)}, with \texttt{size} an 
                integer-valued property of packages indicating their size.
\item[notuptodate($X$)] is the number of packages in $X$ that are not in the latest known version.
\item[unsat\_recommends($X$)] counts the number of disjunctions in Recommends-fields of packages in $X$ that are not satisfied by $S$.
\item[aligned($X$,g1,g2)] is formally $card(\{ (x.g1,x.g2) \mid x \in X \}) - card(\{ x.g1 \mid x \in X \})$.
        In other words, we first cluster the packages in X according to their values at the properties g1 and g2 and count the number of clusters, yielding a value $v1$. Then we do the same when clustering only by the property g1, yielding a value $v2$. The value returned is then $v1-v2$. 
This measure can be used to count \emph{version changes} as defined in~\cite{AlignedUpgrades2011}.
\end{description}
      
\section{Optimising and combining measurements}

The measures described above are used as functions that the solver is then asked to maximise or minimise.
In the preference language, maximisation is specified by prepending the plus sign $+$ to a measure, and
minimisation is specified by prepending a minus sign $-$ to a measure. For example, \texttt{-count(changed)}
requires a solution that minimises changes with respect to the initial configuration, while \texttt{+count(up)}
requests a solution maximising the upgrades.

Several criteria can be aggregated together in lexicographical order, so for
example \texttt{-count(removed),-count(changed)} is a preference expression
that specifies a solution which minimises the number of removed packages,
and then minimises the number of changed ones.

\section{Examples of usage}



\subsection{Examples preferences when installing your packages}
\paragraph{Simple.} Here are two preferences that roughly correspond to a \emph{paranoid} user which is interested in installing a package with minimal changes to the system,
and to a \emph{trendy} user that wants to get the latest and best version of all current packages. They both want to avoid removal of installed packages, but share little less.
\begin{description}
\item[paranoid] \solverpref{-count(removed),-count(changed)}
\item[trendy] \solverpref{-count(removed),-notuptodate(solution),-count(new)}
\end{description}

\paragraph{More refined.} Using the selectors \texttt{request} and \texttt{down}, it is possible to write a more refined version of the above preferences, that may
correspond to the needs of both categories of users: here we ask the solver to get the latest version only for the packages explicitly mentioned in the request.
\begin{description}
\item[paranoidtrendy] \solverpref{-count(removed),-notuptodate(\textcolor{blue}{request}),-count(down),-count(changed)}
\end{description}

\paragraph{Repairing a broken system configuration.} If a system has got into an inconsistent state (package dependencies are broken, conflicts creeped in), it
is quite simple to repair it by issueing an empty request with one of the following user preferences
\begin{description}
\item[fixup simple] \solverpref{-count(changed)}
\item[fixup trendy] \solverpref{-count(changed),-count(down),-notuptodate(solution)}
\end{description}

\paragraph{More exotic examples.} It is also possible to exploit user preferences to obtain a bill of materials to build a system with
specific properties, like a minimal disk footprint (as is done in \url{https://github.com/rdicosmo/ami-thinner} to create minimal Debian
virtual machines), or the largest set of compatible components.

\begin{description}
\item[Minimal system size] \solverpref{-sum(solution,installedsize),-count(solution)}
\item[Noah's ark] \solverpref{+count(solution)}
\item[Noah's ark, fresh] \solverpref{-notuptodate(solution),+count(solution)}
\end{description}

\section{Mixing user request and solver preferences for expressivity}\label{sect:mix}

There may be situations where the precise objective the user is pursuing can be
described by using a combination of the user requestand the solver preferences.

A typical example is the removal of \emph{cruft}, that can accumulate on a
system in the form of packages that ended up installed at some point, to satisfy
dependencies, but that are now no longer needed by any of the packages
explicitly installed by the user (often called \emph{root} packages).  This
functionality is offered for example by package managers like \texttt{apt-get}
or \texttt{aptitude} that provide an \emph{autoremove} command.\\

A package manager built on top of CUDF-compatible dependency solvers can provide
the same functionality by simply specifying in an \emph{install} user request all the
\emph{root} packages, at their current version, and then using the preference
expression \texttt{-count(solution)}. 
%
This will effectively ensure that all root packages stay installed at their
current version, and then minimise the number of extra packages present.\\

As a variant, if one is interested at the same time in getting the latest version
of the root packages, one can use an \emph{upgrade} user request mentioning
all the root packages (that will prevent downgrades of the root packages) and
then use the preference expression \texttt{-notuptodate(request),-count(down),-count(solution)}.
The resulting effect will be to pick a solution that get the latest version
of the root packages, while minimising downgrades and irrelevant extra packages.

\section{Older preference language}

Solvers participating in earlier versions of the MISC competition supported a simpler preference language,
that can be directly translated into the more sophisticated one described above.\\

The correspondence is given below.\\

\begin{center}
\begin{tabular}{|c|c|}
\hline
Old language & New language \\\hline
removed     & count(removed) \\
new         & count(new) \\
changed     & count(changed) \\
notuptodate & notuptodate(solution)\\
unsat\_recommends & unsat\_recommends(solution)\\\hline
\end{tabular}
\end{center}


\section{Solvers supporting user preferences}

At the time of this writing, here is the list of available dependency solvers supporting
the different preference languages:

\begin{description}
 \item[full selectors and measures] \mbox{}\\\texttt{aspcud}\footnote{\url{http://sourceforge.net/projects/potassco/files/aspcud/}} versions \texttt{1.9.1} and later
 \item[MISC 2012 selectors and measures]\mbox{}\\
      \texttt{aspcud} in versions \texttt{1.8} and later;\\
      \texttt{p2cudf} \footnote{\url{http://wiki.eclipse.org/Equinox/p2/CUDFResolver}}
 \item[older preference language]\mbox{}\\
      \texttt{aspcud} versions \texttt{1.9.1} and later,\\
      \texttt{aspcud} versions before \texttt{1.8} (e.g. version \texttt{2011.03.17} in Debian and Ububntu);\\
      \texttt{mccs}\footnote{\url{http://www.i3s.unice.fr/~cpjm/misc/mccs.html}},\\
      and \texttt{packup}\footnote{\url{http://sat.inesc-id.pt/~mikolas/sw/packup/}}.
\end{description}

\subsection{Getting your external solver}
The \texttt{p2cudf} solver is Java based, and sould be quite portable, even if it is not
the fastest among the available solvers.
For Debian and Ubuntu users, the \texttt{mccs}, \texttt{packup} and \texttt{aspcud} solvers
are installable out of the box.

In case one needs a solver on a platform that does not easily accomodate it, either
because a packages version is not available, or because the system has little resources
(like Arduino or RaspberriPi boards), it is now possible to get a solver \emph{in the cloud}.
It is enough to go to \texttt{http://cudf-solvers.irill.org} and follow the instructions therein.

\section{Package managers that can use external dependency solvers}

These solvers can be used by recent versions of the \texttt{apt-get} and \texttt{opam} package managers.

\subsection{Using external solvers with \texttt{apt-get}}

Current versions of \texttt{apt-get} have an option \texttt{--solver} that allows to specify an external
solver. Notice though that because of the inner working of \texttt{apt-get}, one gets better result
by using the wrapper script \texttt{apt-cudf-get} that accepts the same options as apt-get.

\subsection{Using external solvers with \texttt{opam}}
Thanks to the work done by the Mancoosi team at Irill and OCamlPro, in the DORM project,
the \texttt{aspcud} solver is supported out of the box in \texttt{opam} since 1.0.
With a current version of \texttt{opam}, one can use any other CUDF compatible solver:
typing \textcolor{blue}{\texttt{opam --help}} shows 

\begin{alltt}
...
OPTIONS
  \textcolor{blue}{--criteria}=CRITERIA
      Specify user preferences for dependency solving for this run.
      Overrides both $OPAMCRITERIA and $OPAMUPGRADECRITERIA. 
      For details on the supported language, see
      http://opam.ocaml.org/doc/Specifying_Solver_Preferences.html. 
      The default value is \textcolor{blue}{-removed,-notuptodate,-changed} for upgrades,
      and \textcolor{blue}{-removed,-changed,-notuptodate} otherwise.

  \textcolor{blue}{--cudf}=FILENAME
      Debug option: Save the CUDF requests sent to the solver to
      FILENAME-<n>.cudf.

  \textcolor{blue}{--solver}=CMD
      Specify the name of the external dependency solver. The default
      value is \textcolor{blue}{aspcud}
...
\end{alltt}

\section{Proposed selector extension: filters and set operators}\label{sect:filtersel}

We propose to extend the preference language by introducing an additional package selector, \texttt{filter}, and
closing the package selectors by the union, intersection, and difference set operations.

The intended semantics is the following:

\begin{description}
  \item[\texttt{filter(\emph{field} op \emph{value})}] denotes $\{ p \in S ~\mid~  p.field ~op~ value \}$, the set of packages in the solution $S$ whose field \emph{field} contains a value $v$ s.t.
        $v~op~value$. For example  \texttt{filter(root = true)} will select all root packages in the solution (as seen above, in \texttt{opam}, that's the set of packages whose installation was explicitly
        requested by the user, as opposed to packages that are pulled in to satisfy dependencies).
  \item[set operators] have the usual semantics as intersection, union and difference
\end{description}


With filters, it is pretty easy to rewrite the example given in Section~\ref{sect:mix} as the one liner

\begin{center}
\texttt{-count(changed ~and~ filter(root=true)),-count(solution)}
\end{center}

And it is now possible to write a preference expression that roughly means
``do not remove root packages, and bring them to their latest version''

\begin{center}
\texttt{-count(removed ~and~ filter(root=true)),-notuptodate(filter(root=true)}
\end{center}

\section{Conclusions and future work}
User preferences are a powerful tool to orient a dependency solver towards solutions that correspond
to high-level user desiderata. We have presented here the current state of the art for the class of
solvers that did take part in the annual MISC competition over the past years.

More work may be needed to refine and expand the user preferences language, in order to improve its
expressivity, according to concrete use cases, and a concrete proposition for an extension is presented
here.

\nocite{packup-2010, MCCS-2010, aspcud-2011}

%\bibliography{dicosmo,mancoosi}
\bibliography{usercriteria}
\bibliographystyle{abbrv}

\appendix

\section{BNF of the user preference language}

The following BNF grammar describes the preference expression language. Notice that the green nonterminal \aspcudext{\texttt{reqsel}} describes
the package selector extension implemented by \texttt{aspcud}, and the blue part of the \texttt{selector} production correspond to the proposed
extension of the selectors of section~\ref{sect:filtersel}.

\begin{bnf*}
\bnfprod{prefexpr}
{\bnfpn{optim} \bnfor \bnfpn{optim} \bnfsp \bnfts{,} \bnfsp \bnfpn{prefexpr}}
\\
\bnfprod{optim}
{\bnfts{+} \bnfsp \bnfpn{measure} \bnfor \bnfts{-} \bnfsp \bnfpn{measure}}
\\
\bnfprod{measure}
{\bnfpn{unarym} \bnfsp \bnfts{(} \bnfsp  \bnfpn{selector} \bnfsp \bnfts{)} \bnfsp  \bnfor 
 \bnfts{sum}\bnfsp \bnfts{(} \bnfsp  \bnfpn{selector} \bnfsp \bnfts{,} \bnfsp  \bnftd{field} \bnfsp \bnfts{)} \bnfsp  \bnfor
 \bnfts{aligned}\bnfsp \bnfts{(} \bnfsp  \bnfpn{selector} \bnfsp \bnfts{,} \bnfsp  \bnftd{field} \bnfsp \bnfts{,} \bnfsp  \bnftd{field} \bnfsp \bnfts{)} \bnfsp  
}\\
\bnfprod{unarym}
{
  \bnfts{count} \bnfor
  \bnfts{notuptodate} \bnfor
  \bnfts{unsat\_recommends}
}\\
\bnfprod{selector}
{
  \bnfpn{basicsel} \bnfor
  \aspcudext{\bnfpn{reqsel}} \bnfor
  \proposal{
    \bnfpn{filtersel} \bnfor
    \bnfsp \bnfts{(} \bnfsp  \bnfpn{selector} \bnfsp \bnfts{)} \bnfsp  \bnfor
    \bnfpn{selector} \bnfsp \bnfpn{setop} \bnfsp \bnfpn{selector}
  }
}\\
\bnfprod{basicsel}
{\bnfts{new} \bnfor 
  \bnfts{changed} \bnfor
  \bnfts{removed} \bnfor
  \bnfts{up} \bnfor
  \bnfts{down} \bnfor
  \bnfts{solution}
}\\
\bnfprod{reqsel}
{
  \bnfts{request} \bnfor
  \bnfts{installrequest} \bnfor
  \bnfts{upgraderequest}
}\\
\\
\bnfprod{filtersel}
{
  \bnfts{filter} \bnfsp \bnfts{(} \bnfsp  \bnftd{fieldname} \bnfsp \bnfpn{op} \bnfsp \bnftd{value} \bnfsp \bnfts{)} \bnfsp 
}\\
\bnfprod{op}
{
  \bnfts{\texttt{=}}  \bnfor
  \bnfts{\texttt{>}}  \bnfor
  \bnfts{\texttt{<}}  \bnfor
  \bnfts{\texttt{>=}} \bnfor
  \bnfts{\texttt{<=}} \bnfor
  \bnfts{\texttt{<>}}
}\\
\bnfprod{setop}
{
  \bnfts{\texttt{and}} \bnfor
  \bnfts{\texttt{or}} \bnfor
  \bnfts{\texttt{minus}}
}\\
\end{bnf*}


\end{document}
